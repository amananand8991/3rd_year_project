\documentclass[a4paper, 12pt]{article}
\usepackage{bm}
\usepackage{amssymb}
\usepackage{graphicx}
\usepackage{amsmath}
\usepackage{amsfonts}
\usepackage{breqn}
\usepackage{float}
\usepackage{wrapfig}
\graphicspath{ {images/} }
\begin{document}
\begin{center} 
{\Huge{\textbf{Keldysh Field Theory}}}\\
\end{center}

\section {Mathematical Preliminaries}

\subsection {Closed time contour}

The density matrix evolves in time according to the Von Neumann equation as
\begin{equation}%%%%%%%%%% 1 %%%%%%%%
\partial_t \hat{\rho}(t) = -i [\hat{H}(t), \hat{\rho}(t)],
\end{equation}

which can be solved by taking 
\begin{equation}%%%%%%%%%% 2 %%%%%%%%
\hat{\rho}(t) = \hat{\mathcal{U}}_{t,-\infty}\hat{\rho}(-\infty) [\hat{\mathcal{U}}_{t,-\infty}]^{\dagger}.
\end{equation}
Where the unitary evolution operator obeys

\begin{equation}%%%%%%%%%% 3 %%%%%%%%
\partial_t \hat{\mathcal{U}}_{t,t'} = -i \hat{H}(t) \*\ \hat{\mathcal{U}}_{t,t'} ;\quad \partial_{t'} \hat{\mathcal{U}}_{t,t'} = i  \hat{\mathcal{U}}_{t,t'}\*\ \hat{H}(t') .
\end{equation}
The time evolution operator can be written as 
\begin{dmath}
\hat{\mathcal{U}}_{t,t'}  = \lim_{N \to \infty}e^{-i \hat{H}(t-\delta_t)\delta_t}e^{-i \hat{H}(t-2\delta_t)\delta_t}...e^{-i \hat{H}(t')\delta_t}  =\mathbb{T}\mathrm{exp}\bigg(-i \int_{t'}^t \hat{H}(t) dt \bigg).
\end{dmath}
The expectation value of any general operator is given as
\begin{equation}
\langle \hat{\mathcal{O}} \rangle (t) \equiv \frac{\mathrm{Tr}\{\hat{\mathcal{O}}\hat{\rho}(t)\}}{\mathrm{Tr}\{\hat{\rho}(t)\}} = \frac{1}{\mathrm{Tr}\{\hat{\rho}(t)\}} \mathrm{Tr}\{ \hat{\mathcal{U}}_{-\infty,t} \hat{\mathcal{O}} \hat{\mathcal{U}}_{t,-\infty} \hat{\rho}(-\infty)  \}.
\end{equation}
Using the below two equations the general contour can be taken from t = $-\infty$ to t = $\infty$ and again back.
\begin{equation}
\hat{\mathcal{U}}_{t,+\infty}\hat{\mathcal{U}}_{+\infty,t} = \hat{1}, \quad \hat{\mathcal{U}}_{-\infty,t}\hat{\mathcal{U}}_{t,+\infty} = \hat{\mathcal{U}}_{-\infty,+\infty}
\end{equation}
This gives 

\begin{equation}
\langle \hat{\mathcal{O}} \rangle (t)  = \frac{1}{\mathrm{Tr}\{\hat{\rho}(-\infty)\}} \mathrm{Tr}\{\hat{\mathcal{U}}_{-\infty,+\infty} \hat{\mathcal{U}}_{+\infty,t} \hat{\mathcal{O}} \hat{\mathcal{U}}_{t,-\infty} \hat{\rho}(-\infty)  \}.
\end{equation}
We introduce the generating(partition) function as
\begin{equation}
Z[V] \equiv \frac{\mathrm{Tr}\{\hat{\mathcal{U}_C}[V]\hat{\rho}(-\infty)\}}{\mathrm{Tr}\{\hat{\rho}(-\infty)\}}
\end{equation}
where $\hat{\mathcal{U}}_C = \hat{\mathcal{U}}_{-\infty , +\infty} \hat{\mathcal{U}}_{+\infty , -\infty}$ and the hamiltonian for forward and bakward evolution are defined as $\hat{H}^{\pm}_V (t) \equiv \hat{H}(t) \pm \hat{\mathcal{O}}V(t)$. By taking the functional derivatives of the generating function we can calculate
\begin{equation}
\langle \hat{\mathcal{O}} \rangle (t) =(i/2)\frac{\delta Z[V]}{\delta V(t)} \bigg|_{V=0}
\end{equation}


\subsection {Coherent states}
A coherent state parametrized by a complex number $\phi$ is defined as the eigenstate of the annhilation operator  as $\hat{b} |\phi\rangle = \phi |\phi\rangle$. So
\begin{equation}
|\phi\rangle = \sum_{n=0}^{\infty} \frac{\phi^n}{\sqrt{n!}}|n\rangle = e^{\phi \hat{b}^\dagger}|0\rangle.
\end{equation}
It follows that
\begin{equation}
\begin{split}
&\quad \langle\phi|\phi'\rangle = e^{\phi \phi'}, \\
&\quad \hat{1} = \int d[\bar{\phi},\phi]\*\ e^{-|\phi|^2} |\phi\rangle \langle\phi| ,\\
 &\quad Z[\bar{J},J] = \int d[\bar{\phi},\phi]\*\ e^{-\bar{\phi}\phi + \bar{\phi}J + \bar{J}\phi} = e^{\bar{J}J},\\
&\quad \int d[\bar{\phi},\phi]\*\ e^{-|\phi|^2} \bar{\phi}^n\phi^{n'} = \frac{\partial^{n+n'}}{\partial J^n \partial \bar{J}^{n'}} Z[\bar{J},J] \bigg|_{\bar{J}=J=0} = n! \delta_{n,n'},\\
&\quad \mathrm{Tr}\{\mathcal{O}\} = \int d[\bar{\phi},\phi]\*\ e^{-|\phi|^2} \langle\phi|\mathcal{O}|\phi\rangle,\\
&\quad f(\rho) \equiv \langle\phi|\rho^{\hat{b}^{\dagger} \hat{b}}|\phi\rangle = e^{\bar{\phi} \phi' \rho}.
\end{split}
\end{equation}


\section {Bosonic Partition function}
Simplest example of many body system: bosonic particles occupying a single quantum state with energy $\omega_0$. So,
\begin{equation}
\hat{H}(\hat{b}^{\dagger},\hat{b}) = \omega_0 \hat{b}^{\dagger}\hat{b}.
\end{equation}
Choose the initial density matrix be thermal density matrix
\begin{equation}
\hat{\rho}_0 = e^{-\beta(\hat{H}-\mu \hat{N})} = e^{-\beta (\omega_0 -\mu )\hat{b}^{\dagger}\hat{b}}.
\end{equation}
and
\begin{equation}
\mathrm{Tr}\{\hat{\rho}_0\} = \sum_{n=0}^{\infty}e^{-\beta (\omega_0 -\mu )n} = [1 - \rho(\omega_0)]^{-1},
\end{equation}
where $\rho(\omega_0) = e^{-\beta (\omega_0 -\mu )}$. To calculate $\mathrm{Tr}\{\hat{\mathcal{U}}_C \hat{\rho}_0\}$ we divide the time contour $\mathcal{C}$ into 2N parts going from $t=-\infty$ to $+\infty$ and insert the identity in between. So the expression becomes
\begin{equation}
\langle \phi_{2N}|\hat{\mathcal{U}}_{-\delta t}|\phi_{2N-1}\rangle  ... \langle \phi_{N+2}|\hat{\mathcal{U}}_{-\delta t}|\phi_{N+1}\rangle \langle \phi_{N+1}|\hat{1}|\phi_{N}\rangle \langle \phi_{N}|\hat{\mathcal{U}}_{+\delta t}|\phi_{N-1}\rangle ... \langle \phi_{2}|\hat{\mathcal{U}}_{+\delta t}|\phi_{1}\rangle \langle \phi_{1}|\hat{\rho}_0|\phi_{2N}\rangle.
\end{equation} 
Where each of the terms are given as
\begin{equation}
\langle \phi_{j}|\hat{\mathcal{U}}_{\pm \delta t}|\phi_{j-1}\rangle \approx \langle \phi_{j}|(1 \mp i\hat{H}(\hat{b}^{\dagger},\hat{b})\delta t)|\phi_{j-1}\rangle \approx e^{\bar{\phi}_j \phi_{j-1}}e^{\mp  i\hat{H}(\bar{\phi}_j ,\phi_{j-1})\delta t}
\end{equation}











\subsection {Gaussian like integrals}

\subsection {Going Green}


%\section {Keldysh Sorcery}

%\section {Keldysh Kung Fu}

\end{document}